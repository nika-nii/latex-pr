\section{ПРЕДИСЛОВИЕ}

	Настоящая книга является учебным пособием по дисциплине «Методы оптимизации» и воозникла в процессе чтения лекций магистрам специальностей магистрам специальностей «Информатика и вычислительная техника» и «Программная инжерия» Белгородского технологического университета им. В. Г. Шухова. Термин оптимизация можно понимать очень широко, включая в него методику всевозможных компьютерных переборов, лишенных какого-либо математического содержания. Предметом же настоящей книги являются математические методы оптимизации, т. е. математические методы решения экстремальных задач. Первые формулировки таких задач возникли в глубокой древности, а современные методы решения большинства из них стали известны сравнительно недавно. Возникновение и бурное развитие вычислительной техники привело к значительному росту популярности и практической значимости методов оптимизации. В настоящее время любое достаточно серьезное технологическое исследование содержит в себе ту или иную оптимизацию. Поэтому хорошо подготовленный специалист в области информационных технологий должен иметь представление о правильной постановке экстремальных задач и об эффективном выборе алгоритмов их решения. Авторы стремились к популярности изложения, которая, однако, не препятствовала бы полноценному усвоению основных методов и алгоритмов, а также всего курса в целом. Разумеется, охватить все содержание дисциплины в таком пособии невозможно. Здесь представлены лишь основы большинства разделов. Ряд вопросов излагается в обзорном порядке, и даются литературные ссылки для подробного углубленного изучения. Каждая глава снабжена контрольными вопросами и задачами для самостоятельного решения, которые позволяют проверить качество усвоения материала.

	В книге затрагивается широкий круг вопросов, освещавшийся в большом количестве источников. Список литературы, приведенный в конце книги, содержит лишь те публикации, которые были использованы в ней или близко примыкают к ней, дополняя ее содержание.

	Для чтения книги необходимы знания математического анализа, линейной алгебры, аналитической геометрии и теории вероятностей в объеме университетского курса бакалавриата. При рассмотрении вопросов бесконечномерной оптимизации необходимы также некоторые сведения из функционального анализа, которые, однако, кратко изложены в соответствующих местах настоящего пособия.

/section

	Под оптимизацией понимается выбор наилучшего варианта из некоторого множества альтернатив. Постоянное стремление осуществить такой выбор характерно для всех эпох развития человечества. В повседневной жизни задачи оптимизации, как правило, не требуют особых научных методов. Для их решения порой достаточно здравого смысла и накопленного ранее опыта. Однако в более сложных случаях не обойтись без расчетов, использующих математические модели исследуемых объектов. Под математической моделью явления или процесса понимают совокупность формул, уравнений, неравенств и т.д., отражающую существенные черты этого явления или процесса. При создании математических моделей для оптимизации первоочередной задачей является определение параметров, однозначно описывающих исследуемую ситуацию. Эти параметры обычно подразделяют на контролируемые, неконтролируемые и целевые.

%ВСТАВИТЬ ИКСЫ

	\textit{Контролируемые параметры}  являются переменными, принимающими чаще всего числовые значения, причем исследователь может придавать им эти значения по своему усмотрению. 
\textit{Неконтролируемые параметры}  нельзя менять по своему усмотрению; более того, их значения во многих случаях исследователю неизвестны. Они могут быть случайными величинами с известными или неизвестными вероятностными характеристиками. Наконец, \textit{целевые параметры} характеризуют эффективность альтернатив.

	Математическая модель должна связывать контролируемые и неконтролируемые параметры с целевыми параметрами. Задача нахождения по контролируемым и  неконтролируемым параметрам целевых параметров называется \textit{прямой задачей} оценки альтернатив.

	Наряду с прямой задачей часто решают \textit{обратную задачу,} в которой требуется определить такие значения контролируемых параметров, при которых целевые параметры  удовлетворяют тем или иным условиям оптимальности. Такие задачи часто  еще называют \textit{задачами оптимизации.} В зависимости от разновидности и сложности математической модели применяют те или иные методы решения обратной задачи, то есть те или иные \textit{методы оптимизации.} Используемые в настоящее время математические модели подразделяются на детерминированные и стохастические. В детерминированных моделях неконтролируемые параметры, как правило, отсутствуют, а по значениям контролируемых параметров целевые параметры определяются математической моделью однозначно. К детерминированным моделям относятся модели линейного и нелинейного программирования, модели оптимального управления и т.д. Методы оптимизации здесь разработаны достаточно хорошо. В настоящем пособии, в основном, рассмотрены методы оптимизации для детерминированных моделей. В стохастических моделях связь между контролируемыми и целевыми параметрами носит вероятностный характер. Помимо указанного разделения моделей существенную роль играет их разделение по количеству переменных (контролируемых параметров). Говорят о задачах \textit{одномерной, многомерной} и даже \textit{бесконечномерной оптимизации.}

	Первые две главы настоящего пособия посвящены методам решения задач линейного программирования. В главах 3 и 4 кратко излагается теория двойственности линейного программирования и ее приложения в теории игр. Глава 5 посвящена методам решения задач дискретного линейного программирования. Основам нелинейного программирования посвящена глава 6, в которой изложен классический подход к теории экстремумов и общие численные методы решения задач. Здесь рассмотрены также задачи выпуклого и квадратичного выпуклого программирования. Задачам динамического программирования посвящена глава 7. Все перечисленные главы освещают вопросы конечномерной оптимизации. Бесконечномерной оптимизации посвящена последняя восьмая глава. Здесь мы кратко описываем основы вариационного исчисления и методы решения вариационных задач. Особое внимание уделяется прямым методам вариационного исчисления. Отметим, что в главе 8 рассмотрены лишь некоторые классические общие прямые методы вариационного исчисления. Существует большое число прямых методов, разработанных для решения тех или иных частных задач. В добавлении в качестве примера изложен прямой метод решения задачи об оптимальном распределении источников тепла.

\chapter{ОСНОВНЫЕ ПОНЯТИЯ И МЕТОДЫ ЛИНЕЙНОГО ПРОГРАММИРОВАНИЯ}

В 1939 году Леонид Витальевич Канторович опубликовал работу «Математические методы организации и планирования производства», в которой сформулировал новый класс экстремальных задач с ограничениями и разработал эффективный метод их решения. Тем самым были заложены основы линейного программирования. Джордж Данциг разработал симплекс метод и считается «отцом линейного программирования» на западе. Слово программирование здесь означает составление оптимального плана (программы) производства.

\subsection{Формулировка задач линейного программирования. Основные формы линейных моделей}

Сущность линейного программирования состоит в нахождении точек наибольшего или наименьшего значения некоторой линейной функции при определенном наборе линейных ограничений, налагаемых на аргументы. Ограничения образуют \textit{систему ограничений}, которая имеет, как правило, бесконечное множество решений. Каждая совокупность значений переменных (аргументов функции z), которые удовлетворяют системе ограничений, называется \textit{допустимым планом} задачи линейного программирования. Функция z, максимум или минимум которой определяется, называется \textit{целевой функцией задачи}. Допустимый план, на котором достигается  максимум или минимум функции z, называется \textit{оптимальным планом задачи.} Система ограничений, определяющая множество планов, диктуется условиями производства. Задача линейного программирования состоит в выборе из множества допустимых планов наиболее выгодного (оптимального).

\bfseries{Пример 1.1.} Мастерская имеет два станка 


