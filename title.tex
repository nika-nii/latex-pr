\thispagestyle{empty}
\begin{center}
\textbf{МИНИСТЕРСТВО ОБРАЗОВАНИЯ И НАУКИ \\ РОССИЙСКОЙ ФЕДЕРАЦИИ \\
Белгородский государственный технологический университет \\ имени В. Г. Шухова}
\end{center}
\vspace{1.5cm}
\begin{center}
\large\textbf{{А. Г. Брусенцев, О. В. Осипов}}
\end{center}
\vspace{1.5cm}
\begin{center}
\large\textbf{{МЕТОДЫ ОПТИМИЗАЦИИ}}
\vspace{0.3cm}

\textbf{Учебное пособие}
\end{center}
\vspace{9.5cm}
\begin{center}
\textbf{Белгород \\
2018}
\end{center}
%%%%%%%%%
\newpage
%%%%%%%%%
\begin{center}
МИНИСТЕРСТВО ОБРАЗОВАНИЯ И НАУКИ \\ РОССИЙСКОЙ ФЕДЕРАЦИИ \\
Белгородский государственный технологический университет
имени В. Г. Шухова
\end{center}
\vspace{1.5cm}
\begin{center}
А. Г. Брусенцев, О. В. Осипов
\end{center}
\vspace{1cm}
\begin{center}
МЕТОДЫ ОПТИМИЗАЦИИ
\vspace{0.3cm}

\footnotesize{Утверждено учёным советом университета в качестве учебного \\ пособия для студентов направлений подготовки \\
09.04.01 - Информатика и вычислительная техника, \\ 09.04.04 - Программная инженерия}
\end{center}
\vspace{9cm}
\begin{center}
Белгород \\
2018
\end{center}
%%%%%%%%%%
\newpage
%%%%%%%%%%
\begin{flushleft}
УДК 519.8 \\
ББК 22.1 \\
Б89
\end{flushleft}
\vspace{0.3cm}
\begin{center}
Рецензенты:
\end{center}
\begin{center}
\small{Доктор технических наук, профессор Белгородского государственного технологического университета им. В. Г. Шухова  \textbf{\textit{Г. М. Редькин}}
Кандидат физико-математических наук, доцент Белгородского государственного национального исследовательского университета \\
(НИУ «БелГУ»)  \textbf{\textit{В. В. Флоринский}}}
\end{center}
\vspace{1.6cm}
\justify{\textbf{Брусенцев А. Г}}
\begin{center}
Б89	Методы оптимизации: учебное пособие / А. Г. Брусенцев, \\
О. В. Осипов. – Белгород: Изд-во БГТУ, 2018. – 263 с.%TODO: количество страниц
\end{center}
\vspace{1.1cm}
\footnotesize{В пособии рассматриваются методы решения задач оптимизации. Изложены основы линейного программирования, теории двойственности и теории игр. Представлены различные методы дискретного, нелинейного и динамического программирования. Рассматриваются методы бесконечномерной оптимизации: классические и прямые методы вариационного исчисления. Изложение иллюстрируется примерами. В каждой главе приведены контрольные вопросы и задачи для самостоятельной работы.

Издание будет способствовать полноценному усвоению основных методов и алгоритмов оптимизации.

Учебное пособие предназначено для студентов высших учебных заведений, обучающихся по направлениям 09.04.01 – Информатика и вычислительная техника и 09.04.04 – Программная инженерия, а также может быть использовано и студентами других инженерно-экономических специальностей, изучающих дисциплину «Методы оптимизации».

Публикуется в авторской редакции.}
\begin{flushright}
\vspace{0.3cm}
\textbf{УДК 519.8 \\
ББК 22.1\\}
\vspace{0.3cm}
\textcopyright\,Белгородский государственный \\
технологический университет \\
(БГТУ) им. В. Г. Шухова, 2018
\end{flushright}
