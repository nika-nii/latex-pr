\begin{thebibliography}{99}
% 1
\bibitem{literature_akulich}
Акулич, И.Л.  Математическое программирование в примерах и задачах / И.Л. Акулич. — М.: Высш. шк. ,1986. — 318 с.
% 2
\bibitem{literature_ahiezer}
Ахиезер Н.И.  Вариационное исчисление / Н.И. Ахиезер. – Харьков: Вища школа. – Изд-во при Харьк. ун-те, 1981.– 168с.
% 3
\bibitem{literature_ashmanov_1981}
Ашманов, С.А. Линейное программирование / С.А. Ашманов. — М.: Наука, Главная редакция физико-математической литературы, 1981.— 340 с.
% 4
\bibitem{literature_ashmanov_1991}
Ашманов, С.А. Теория оптимизации в задачах и упражнениях / С.А. Ашманов, А.В. Тихонов. — М.: Наука, 1991. — 447 с
% 5
\bibitem{literature_bellman}
Беллман, Р.  Динамическое программирование / Р Беллман. — М.: ИЛ, 1960. — 430 с.
% 6
\bibitem{literature_boltyansky}
Болтянский, В.Г.  Оптимальное управление дискретными системами / В.Г. Болтянский. — М.: Наука, 1973. — 446 с.
% 7
\bibitem{literature_brusencev_2010}
Брусенцев А.Г., Брусенцева В.С.  Задача об оптимальном выборе источников тепла. // Cб. трудов XXIII международной конференции «Математические методы в технике и технологиях». – Т.2. – 2010. – С. 43–46.
% 8
\bibitem{literature_brusencev_2013}
Брусенцев А.Г., Осипов О.В.  Оптимальный выбор источников тепла при наличии конвекции // Научные ведомости БелГУ. Математика. Физика. – № 26 (169). Выпуск 33. Белгород. – 2013. – С. 64–82.
% 9
\bibitem{literature_brusencev_2012}
Брусенцев А.Г., Осипов О.В.  Приближенное решение задачи об оптимальном выборе источников тепла // Научные ведомости Белгородского государственного университета. --  №5 (124). Выпуск 26. Белгород. – 2012. – С. 60–69.
% 10
\bibitem{literature_brusencev_2016}
Брусенцев А.Г., Осипов О.В.  Численное нахождение обменной матрицы при решении задачи оптимизации распределения источников тепла // Вестник Белгородского государст-венного технологического университета имени В.Г. Шухова. №5. – 2016. – С.116-124.
% 11
\bibitem{literature_vagner}
Вагнер, Г. Основы исследования операций / Г. Вагнер. — М.: Мир, 1972 — 1973. Т.1 — 3. — 987 c.
% 12
\bibitem{literature_ventzel}
Вентцель, Е.С. Исследование операций (Задачи, принципы, методология) / Е.С. Вентцель. — М: Наука, 1980. — 208 с.
% 13
\bibitem{literature_volkov}
Волков, И.К. Исследование операций / И.К. Волков, Е.А. Загоруйко. — М.: Изд. МГТУ им. Н.Э. Баумана, 2004. — 440 с.
% 14
\bibitem{literature_gladkov}
Гладков Л. А., Курейчик В. В, Курейчик В. М. Биоинспирированные методы в оптимизации: монография / Л. А. Гладков, В. В Курейчик, В. М. Курейчик и др. —М.: Физматлит, 2009. — 384 с.
% 15
\bibitem{literature_golshtein_1}
Гольштейн, Е.Г. Задачи линейного программирования транспортного типа / Е.Г. Гольштейн, Д.Б. Юдин. — М.: Наука, 1969. — 382 с.
% 16
\bibitem{literature_golshtein_2}
Гольштейн, Е.Г. Линейное программирование / Е.Г. Гольштейн, Д.Б. Юдин. — М.: Наука, 1969. — 387 с.
% 17
\bibitem{literature_danzig}
Данциг, Дж. Линейное программирование, его обобщения и приложения / Дж. Данциг. — М.: Прогресс, 1966. — 600 с.
% 18
\bibitem{literature_dikin}
Дикин, И.И. Метод внутренних точек в линейном и нелинейном программировании / И.И. Дикин. — Изд. группа URSS, 2010. —120 с.
% 19
\bibitem{literature_zaychenko}
Зайченко, Ю.П. Исследование операций / Ю.П. Зайченко. — Киев: Выща школа, 1988. — 550с.
% 20
\bibitem{literature_zaslavsky}
Заславский, Ю.Л. Сборник задач по линейному программированию / Ю.Л. Заславский. — М.: Наука, 1969. — 256с.
% 21
\bibitem{literature_kremer}
Исследование операций в экономике / под редакцией профессора Н.Ш. Кремера. — М.: ЮНИТИ, 2003. — 407с.
% 22
\bibitem{literature_kalihman}
Калихман, И.Л. Сборник задач по математическому программированию / И.Л. Калихман . — М.: Высш. шк., 1975. —270 с.
% 23
\bibitem{literature_kantorovich}
Канторович Л.В., Крылов В.И. Приближенные методы высшего анализа / Л.В. Канторович, В.И. Крылов. – М.: Физматгиз.–1962.– 709с.
% 24
\bibitem{literature_karpelevich}
Карпелевич, Ф.И. Элементы линейной алгебры и линейного программирования / Ф.И. Карпелевич, Л.Е. Садовский. — М.: Наука, 1967. — 274 с.
% 25
\bibitem{literature_krushevsky}
Крушевский, А.В. Теория игр / А.В. Крушевский. — Киев: Издательское объединение «Вища школа», 1977. — 216 с.
% 26
\bibitem{literature_kuznetsov}
Кузнецов, Ю.Н. Математическое программирование / Ю.Н. Кузнецов,  В.И. Кузубов,  А.Б. Волощенко. — М.: Высш. шк., 1980. — 371 с.
% 27
\bibitem{literature_lyashenko}
Линейное и нелинейное программирование / под редакцией профессора И.Н. Ляшенко. — Киев: Издательское объединение «Вища школа», 1975. — 370с.
% 28
\bibitem{literature_mainika}
Майника, Э. Алгоритмы оптимизации на сетях и графах: Пер. с англ. / Э. Майника. — М.: Мир, 1981. — 323 с.
% 29
\bibitem{literature_mihlin_1970}
Михлин С.Г. Вариационные методы в математической физике / С.Г. Михлин.– М.: Наука, – 1970.– 512с.
% 30
\bibitem{literature_mihlin_1952}
Михлин С.Г. Проблема минимума квадратичного функционала / С.Г. Михлин. – М.: Гостехиздат.– 1952.– 218с.
% 31
\bibitem{literature_mihlin_1966}
Михлин С.Г. Численная реализация вариационных методов / С.Г. Михлин. – М.: Наука, – 1966.– 432с.
% 32
\bibitem{literature_morozov}
Морозов, В.В. Исследование операций в задачах и упражнениях / В.В. Морозов,  А.Г. Сухарев,  В.В. Федоров. — М: Высш. шк., 1986. — 314 с.
% 33
\bibitem{literature_myshkis}
Мышкис А.Д. Математика. Специальные курсы для ВТУЗов / А.Д. Мышкис.– М.:Наука.– 1971.–632с.
% 34
\bibitem{literature_neiman}
Нейман, Дж. Теория игр и экономическое поведение / Дж. Нейман, О. Моргерштерн. — М.: Наука, 1970. — 708 с.
% 35
\bibitem{literature_osipov}
Осипов О.В. Оптимальное расположение источников тепла в неоднородной среде // Вестник Белгородского государственного технологического университета имени В.Г. Шухова. №1. – 2013. – С. 154–158.
% 36
\bibitem{literature_saati}
Саати, Т.Л.  Математические методы исследования операций / Т.Л. Саати. — М.: Воениздат, 1963. — 353 с.
% 37
\bibitem{literature_efimov}
Сборник задач по математике для ВТУЗов / под редакцией А.В. Ефимова и А.С. Поспелова. – М.: Изд-во Физико-матем. литер. – Т.3. – 2003.– 575с.
% 38
\bibitem{literature_skhreiver}
Схрейвер, А. Теория линейного и целочисленного программирования: В 2-х т.: Пер с англ. / А. Схрейвер. — М.: Мир, 1991. — 360 с, 342 с.
% 39
\bibitem{literature_taha}
Таха, Х. Введение в исследование операций / Х. Таха. — Изд. Вильямс, 2005. — 903c.
% 40
\bibitem{literature_tihonov}
Тихонов, А.Н. Методы решения некорректных задач / А.Н. Тихонов, В.Я. Арсенин. — М.: Наука, 1974. — 222с.
% 41
\bibitem{literature_ford}
Форд, Л. Р.  Потоки  в  сетях:  Пер.  с  англ. / Л. Р. Форд,  Д. Р. Фалкерсон. — М.: Мир, 1966. — 276 с.
% 42
\bibitem{literature_hachiyan}
Хачиян, Л.Г. Полиномиальный алгоритм в линейном программировании  /  Л. Г.  Хачиян.  //  ЖВМ  и  МФ.  —  1980.  — т. 20. — №1, с. 51 — 68.
% 43
\bibitem{literature_hedli}
Хедли, Дж. Нелинейное и динамическое программирование / Дж. Хедли. — М.: Мир, 1967. — 560 с.
\end{thebibliography}
