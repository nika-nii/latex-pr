\section{ОСНОВНЫЕ ПОНЯТИЯ И МЕТОДЫ ЛИНЕЙНОГО ПРОГРАММИРОВАНИЯ}

В 1939 году Леонид Витальевич Канторович опубликовал работу «Математические методы организации и планирования производства», в которой сформулировал новый класс экстремальных задач с ограничениями и разработал эффективный метод их решения. Тем самым были заложены основы линейного программирования. Джордж Данциг разработал симплекс метод и считается <<отцом линейного программирования>> на западе. Слово программирование здесь означает составление оптимального плана (программы) производства.

\subsection{Формулировка задач линейного программирования. Основные формы линейных моделей }

В ограниченной связной области $D{\subset}R^m$ требуется определить функцию $f(\vec x)\! \geqslant\! 0$, доставляющую минимум линейному функционалу
\begin{equation}
\label{equation_2_1}
	J\{f\} = \int\limits_D f(\vec x)\,dV_m \to \min,
\end{equation}
при следующих условиях
\begin{equation}
\label{equation_2_2}
	\begin{split}
		&\chi\Delta u + f = 0,\\
		&\left.\left(\chi\frac{\partial u}{\partial n} + \alpha u\right)\right|_{{\partial}D} = 0,
	\end{split}
\end{equation}
\begin{equation}
\label{equation_2_3}
	M(\vec x) - T_0 \geqslant u(\vec x) \geqslant m(\vec x) - T_0.
\end{equation}

В задаче~(\ref{equation_2_1}),~(\ref{equation_2_2}) можно варьировать...


\subsection{Комбинаторные свойства циклов в матрице}
Под матрицей в этом параграфе понимается таблица, состоящая из клеток. Циклом в матрице будем называть ломаную линию...


В программе все входные данные для расчёта вводятся в таблицу <<Параметры>>, в которой можно задавать следующие функции и константы (табл. \ref{table_2_1}).


\begin{table}[h!]
\caption{Обозначения функций и коэффициентов в таблице <<Параметры>>}
\label{table_2_1}
\begin{tabular}[t]{|p{4em}|p{6em}|p{15em}|p{4em}|}
\hline
 {Обозна-чение} & Область определения & Название параметра &	Размер-ность \\ \hline
\multicolumn{1}{|c|} { $m$ } & { $D$ } & Минимальный профиль
температуры &	Кельвин \\ \hline
\multicolumn{1}{|c|} { $M$ } & { $D$ } & Максимальный профиль
температуры &	Кельвин \\ \hline
\multicolumn{1}{|c|} { $T_0$ } & { Внешняя среда} & Температура 
внешней среды &	Кельвин \\ \hline
\multicolumn{1}{|c|} { $\nu$ } & {$\partial D$} & Скорость движения вдоль нормали к границе & м/с \\ \hline
\multicolumn{1}{|c|} {$\chi$} & {$D$} &	Температуропроводность &	{м$^2$/с} \\ \hline
\multicolumn{1}{|c|} {$\alpha$} & {$\partial D$} &  Функция теплопередачи через границу во внешнюю среду	& м/с \\ \hline
\end{tabular}
\end{table} 


\begin{table}[h!]
\caption{Обозначения функций}
\label{table_2_2}
\begin{tabular}[t]{|p{4em}|p{6em}|p{15em}|p{4em}|}
\hline
 {Обозна-чение} & Область определения & Название параметра &	Размер-ность \\ \hline
\multicolumn{1}{|c|} { $m$ } & { $D$ } & Минимальный профиль
температуры &	Кельвин \\ \hline
\multicolumn{1}{|c|} { $M$ } & { $D$ } & Максимальный профиль
температуры &	Кельвин \\ \hline
\end{tabular}
\end{table} 


\begin{figure}[h!]
	\begin{center}
		\includegraphics[width=7cm]{pictures/ch_3_flowchart}
		\caption{Блок-схема алгоритма решения $m$-мерной задачи}
 	 \label{picture_2_1}
	\end{center}
\end{figure}

\teorema{При сдвиге по означенному циклу на число x решение системы ограничений транспортной задачи переходит снова в решение этой же системы ограничений.}

\teorema{Для каждой свободной клетки опорного решения системы ограничений транспортной задачи существует цикл пересчета и притом только один. }


\teorema{При сдвиге по означенному циклу на число x решение системы ограничений транспортной задачи переходит снова в решение этой же системы ограничений.}

\teorema{Для каждой свободной клетки опорного решения системы ограничений транспортной задачи существует цикл пересчета и притом только один. }

\lemma {Во всяком означенном цикле число положительных вершин, лежащих в каждой строке или столбце, равняется количеству отрицательных вершин в этой строке или столбце.}

\lemma {Если матрица перевозок является базисным решением системы ограничений транспортной задачи, то не существует цикла с вершинами только в базисных клетках.}

{\bf {Задания}}

\zadanie {Для изготовления трех видов изделий $A$, $B$, $C$ используется токарное, фрезерное, сварочное и шлифовальное оборудование. Затраты времени на обработку одного изделия для каждого из типов оборудования указаны в следующей ниже таблице. В ней же указан общий фонд рабочего времени каждого из типов оборудования, а также прибыль от реализации одного изделия каждого вида.
Требуется определить, сколько изделий каждого вида следует изготовить предприятию, чтобы прибыль от их реализации была максимальной.}

\zadanie {Кондитерская фабрика для производства трех видов карамели $A$, $B$ и $C$ использует три вида основного сырья: сахарный песок, патоку и фруктовое пюре. Нормы расхода сырья каждого вида на производство 1 т. карамели данного вида приведены в таблице. В ней же указано общее количество сырья каждого вида, которое может быть использовано фабрикой, а также приведена прибыль от реализации 1 т. карамели каждого вида.}


\opredelenie {Локальным минимумом задачи называется такая точка, удовлетворяющая системе ограничений }

\opredelenie { Задача на минимум нелинейного программирования называется одноэкстремальной, если каждый ее локальный минимум одновременно является и глобальным. Аналогично понимается одноэкстремальность задачи на максимум. }

\zamechanie {При оценивании результатов приближенного решения задачи...}