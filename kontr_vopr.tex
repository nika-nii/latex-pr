\documentclass{article}

\usepackage[russian]{babel}
\usepackage[utf8]{inputenc} 

\begin{document}
\subsection*{Контрольные вопросы и задачи для самостоятельного решения}
\begin {enumerate}
\item Как формулируется общая задача линейного программирования?
\item Когда задача линейного программирования называется имеющей каноническую форму?
\item Какая форма задачи линейного программирования называется стандартной?
\item В чем заключается геометрическое истолкование системы ограничений и целевой функции задачи в случае двух переменных?
\item Дайте определения базисного вида системы линейных уравнений, базисного и опорного решений такой системы.
\item Сформулируйте фундаментальную теорему симплекс-метода.
\item К какому виду должна быть приведена задача линейного программирования перед применением симплекс-метода?
\item Как составить первую симплекс-таблицу?
\item Опишите порядок работы с симплекс-таблицей. В чем заключается признак того, что симплекс-таблица является последней? Как прочесть решение задачи по последней симплекс-таблице? В каком случае по последней симплекс-таблице можно заключить, что задача не имеет решения по причине неограниченности целевой функции на области допустимых значений?
\item Для чего применяется метод искусственного базиса? Какие основные случаи могут представиться при работе этим методом?
\item Опишите метод больших штрафов. Как составить M-задачу для задачи линейного программирования в канонической форме?
\item Как избежать зацикливания симплекс алгоритма?
\item Что понимается под трудоемкостью симплекс метода? Что означает его экспоненциальная трудоемкость на классе всех задач линейного программирования?
\item Существуют ли алгоритмы решения задач линейного программирования полиномиальной трудоемкости? Обладает ли класс всех задач линейного программирования полиномиальной сложностью?
\end{enumerate}

\textit{Построить математические модели в задачах 1.1 — 1.4.}
%1.1-1.3 добавить
1.4. При производстве чугунного литья используется n различных исходных шихтовых материалов (чугун различных марок, стальной лом, феррофосфор и др.) Химический состав чугунного литья определяется содержанием в нем химических элементов (кремния, марганца, фосфора и др.). Готовый чугун должен иметь строго определенный химический состав, который определяется величинами $H_j$, представляющими собой доли (в процентах) $j$-го химического элемента в готовом продукте $(j=1,2,\ldots,m)$. При этом считаются известными величины $h_{ij}$ — содержание (в процентах) $j$-го химического элемента в $i$-ом исходном шихтовом материале, а также величины $c_i$ — цены единицы каждого шихтового материала $(i=1,2,\ldots,n)$. Определить состав шихты, обеспечивающий получение литья заданного качества при минимальной общей стоимости используемых шихтовых материалов.


\textit{В задачах 1.5 — 1.8 привести математическую модель линейного программирования к каноническому виду.}
\[z=-2x_1 x_2 + x_3 \rightarrow min,\]
$$
\left\{
\begin{array}{ccccc}
2x_1 &-x_2 &+6x_3 &\leq &12, \\
3x_1 &+5x_2 &-12x_3 &= &14, \\
-3x_1 &+6x_2 &-4x_3 &\leq &18,
\end{array}
\right.
$$
\[x_1, x_2, x_3\geq 0.\]

\end{document}