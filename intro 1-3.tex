\section{ПРЕДИСЛОВИЕ}
	Настоящая книга является учебным пособием по дисциплине «Методы оптимизации» и воозникла в процессе чтения лекций магистрам специальностей магистрам специальностей «Информатика и вычислительная техника» и «Программная инжерия» Белгородского технологического университета им. В. Г. Шухова. Термин оптимизация можно понимать очень широко, включая в него методику всевозможных компьютерных переборов, лишенных какого-либо математического содержания. Предметом же настоящей книги являются математические методы оптимизации, т. е. математические методы решения экстремальных задач. Первые формулировки таких задач возникли в глубокой древности, а современные методы решения большинства из них стали известны сравнительно недавно. Возникновение и бурное развитие вычислительной техники привело к значительному росту популярности и практической значимости методов оптимизации. В настоящее время любое достаточно серьезное технологическое исследование содержит в себе ту или иную оптимизацию. Поэтому хорошо подготовленный специалист в области информационных технологий должен иметь представление о правильной постановке экстремальных задач и об эффективном выборе алгоритмов их решения. Авторы стремились к популярности изложения, которая, однако, не препятствовала бы полноценному усвоению основных методов и алгоритмов, а также всего курса в целом. Разумеется, охватить все содержание дисциплины в таком пособии невозможно. Здесь представлены лишь основы большинства разделов. Ряд вопросов излагается в обзорном порядке, и даются литературные ссылки для подробного углубленного изучения. Каждая глава снабжена контрольными вопросами и задачами для самостоятельного решения, которые позволяют проверить качество усвоения материала.

	В книге затрагивается широкий круг вопросов, освещавшийся в большом количестве источников. Список литературы, приведенный в конце книги, содержит лишь те публикации, которые были использованы в ней или близко примыкают к ней, дополняя ее содержание.

	Для чтения книги необходимы знания математического анализа, линейной алгебры, аналитической геометрии и теории вероятностей в объеме университетского курса бакалавриата. При рассмотрении вопросов бесконечномерной оптимизации необходимы также некоторые сведения из функционального анализа, которые, однако, кратко изложены в соответствующих местах настоящего пособия.

/section{ВВЕДЕНИЕ}
	Под оптимизацией понимается выбор наилучшего варианта из некоторого множества альтернатив. Постоянное стремление осуществить такой выбор характерно для всех эпох развития человечества. В повседневной жизни задачи оптимизации, как правило, не требуют особых научных методов. Для их решения порой достаточно здравого смысла и накопленного ранее опыта. Однако в более сложных случаях не обойтись без расчетов, использующих математические модели исследуемых объектов. Под математической моделью явления или процесса понимают совокупность формул, уравнений, неравенств и т.д., отражающую существенные черты этого явления или процесса. При создании математических моделей для оптимизации первоочередной задачей является определение параметров, однозначно описывающих исследуемую ситуацию. Эти параметры обычно подразделяют на контролируемые, неконтролируемые и целевые.

%ВСТАВИТЬ ИКСЫ

	{Контролируемые параметры}  являются переменными, принимающими чаще всего числовые значения, причем исследователь может придавать им эти значения по своему усмотрению. 
{Неконтролируемые параметры}  нельзя менять по своему усмотрению; более того, их значения во многих случаях исследователю неизвестны. Они могут быть случайными величинами с известными или неизвестными вероятностными характеристиками. Наконец, {целевые параметры} характеризуют эффективность альтернатив.

	Математическая модель должна связывать контролируемые и неконтролируемые параметры с целевыми параметрами. Задача нахождения по контролируемым и  неконтролируемым параметрам целевых параметров называется {прямой задачей} оценки альтернатив.

	Наряду с прямой задачей часто решают {обратную задачу,} в которой требуется определить такие значения контролируемых параметров, при которых целевые параметры  удовлетворяют тем или иным условиям оптимальности. Такие задачи часто  еще называют 