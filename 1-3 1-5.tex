\documentclass[12pt]{article}
\usepackage[russian]{babel}
\usepackage[utf8]{inputenc} 
\begin{document}






\subsection{Общие системы линейных уравнений. Базисный вид системы. Метод Гаусса-Жордана}
Ниже нам понадобятся некоторые сведения о системах уравнений, состоящих из m линейных уравнений с \textit {n} неизвестными $x_1$,$x_2$,\dots,$x_\textit{n}$%
% ТУТ ЕСТЬ СЛАУ
Такая система может иметь множество решений, состоящее из единственного решения или бесконечного количества решений. Возможен также случай \textit{несовместности системы}, когда множество решений является пустым. Две системы называются\textit {эквивалентными}, если они имеют одинаковые множества решений. Следующие операции над системой приводят к новой системе, эквивалентной исходной:
		\begin{enumerate}
			\renewcommand{\theenumi}{(\arabic{enumi})}
			\renewcommand{\labelenumi}{\arabic{enumi})}
			\itemперестановка уравнений системы;
			\itemумножение обеих частей уравнения системы на одно и то же отличное от нуля число;
			\itemприбавление к заданному уравнению системы любого другого уравнения, умноженного на произвольное число.
		\end{enumerate}
Эти операции называют элементарными операциями над системой.

Основной задачей теории линейных систем является задача нахождения всего множества решений системы или установления ее несовместности. Для ее решения нужно привести систему к эквивалентной системе, имеющей особый \textit{базисный вид}. При этом систему условно записывают в виде расширенной матрицы системы
%ТУТ ЕСТЬ МАТРИЦА
Система называется имеющей базисный вид, если среди столбцов коэффициентов при неизвестных в ее расширенной матрице имеется столько  различных  единичных  столбцов, сколько  ненулевых строк в этой матрице. Единичным столбцом мы называем столбец, в котором на некотором месте стоит единица, а на всех остальных местах — нули. Единичные столбцы считаются различными, если единицы у них находятся на различных местах. Неизвестные, отвечающие этим различным единичным столбцам, называются \textit{базисными неизвестными системы}. Остальные неизвестные называются \textit{свободными неизвестными}. Базисные неизвестные входят по одному в каждое из уравнений и легко выражаются через свободные неизвестные. Если свободным неизвестным придавать произвольные значения, то базисные неизвестные по ним определятся однозначно, и мы получим все решения системы.














\end{document}
