\documentclass{article}

\usepackage[russian]{babel}
\usepackage[utf8]{inputenc} 

\begin{document}






\subsection{Общие системы линейных уравнений. Базисный вид системы. Метод Гаусса-Жордана}
Ниже нам понадобятся некоторые сведения о системах уравнений, состоящих из m линейных уравнений с \textit {n} неизвестными $x_1$,$x_2$,\dots,$x_\textit{n}$%
% ТУТ ЕСТЬ СЛАУ

Такая система может иметь множество решений, состоящее из единственного решения или бесконечного количества решений. Возможен также случай \textit{несовместности системы}, когда множество решений является пустым. Две системы называются\textit {эквивалентными}, если они имеют одинаковые множества решений. Следующие операции над системой приводят к новой системе, эквивалентной исходной:
		\begin{enumerate}
			\renewcommand{\theenumi}{(\arabic{enumi})}
			\renewcommand{\labelenumi}{\arabic{enumi})}
			\itemперестановка уравнений системы;
			\itemумножение обеих частей уравнения системы на одно и то же отличное от нуля число;
			\itemприбавление к заданному уравнению системы любого другого уравнения, умноженного на произвольное число.
		\end{enumerate}
Эти операции называют элементарными операциями над системой.

Основной задачей теории линейных систем является задача нахождения всего множества решений системы или установления ее несовместности. Для ее решения нужно привести систему к эквивалентной системе, имеющей особый \textit{базисный вид}. При этом систему условно записывают в виде расширенной матрицы системы
%ТУТ ЕСТЬ МАТРИЦА
Система называется имеющей базисный вид, если среди столбцов коэффициентов при неизвестных в ее расширенной матрице имеется столько  различных  единичных  столбцов, сколько  ненулевых строк в этой матрице. Единичным столбцом мы называем столбец, в котором на некотором месте стоит единица, а на всех остальных местах — нули. Единичные столбцы считаются различными, если единицы у них находятся на различных местах. Неизвестные, отвечающие этим различным единичным столбцам, называются \textit{базисными неизвестными системы}. Остальные неизвестные называются \textit{свободными неизвестными}. Базисные неизвестные входят по одному в каждое из уравнений и легко выражаются через свободные неизвестные. Если свободным неизвестным придавать произвольные значения, то базисные неизвестные по ним определятся однозначно, и мы получим все решения системы.

\textbf{Пример 1.4.} Рассмотрим систему
%ТУТ ЕСТЬ СЛАУ

Ее расширенная матрица
%ТУТ ЕСТЬ МАТРИЦА

Система имеет базисный вид. Базисными неизвестными будут $x_1$ и $x_4$, а свободными — $x_2$ и $x_3$.
%ТУТ ЕСТЬ СЛАУ

Полагая, свободные неизвестные произвольными $x_2=c_1$, $x_3=c_2$, получаем двухпараметрическое семейство всех решений нашей системы ($6-3c_1-5c_2;c_1;c_2;-1-7c_1+2c_2$).

Свободные неизвестные могут отсутствовать в базисном виде системы. Тогда, очевидно, система имеет только одно решение.

Метод Гаусса-Жордана представляет собой некоторый алгоритм, приводящий систему к базисному виду с помощью цепочки элементарных преобразований, которую удобно выполнять не над системой, а над ее расширенной матрицей. При этом элементарные операции над системой становятся следующими операциями над расширенной матрицей:
		\begin{enumerate}
			\renewcommand{\theenumi}{(\arabic{enumi})}
			\renewcommand{\labelenumi}{\arabic{enumi})}
			\itemперестановка строк в матрице;
			\itemумножение всех элементов некоторой строки на одно и то же отличное от нуля число;
			\itemприбавление к данной строке любой другой, умноженной на произвольное число.
		\end{enumerate}

Метод Гаусса-Жордана состоит из ряда однотипных шагов. Опишем первый шаг алгоритма. Он состоит из трех этапов: 
		\begin{enumerate}
			\renewcommand{\theenumi}{(\arabic{enumi})}
			\renewcommand{\labelenumi}{\arabic{enumi})}
			\itemсреди коэффициентов при неизвестных расширенной матрицы системы выбирается отличное от нуля число, которое в дальнейшем мы называем  \textit{разрешающим элементом} шага метода;
			\itemстрока разрешающего элемента делится на разрешающий элемент и полученная строка, становясь основным инструментом для преобразования матрицы, называется нами в дальнейшем \textit{ведущей строкой} шага алгоритма;
			\itemведущая строка преобразует остальные строки матрицы путем прибавления ее к этим строкам после умножения на так подобранные числа, чтобы после преобразований в столбце бывшего разрешающего элемента стояли нули на всех местах, кроме места самого разрешающего элемента (на котором находится единица).
		\end{enumerate}

Описанные преобразования являются цепочкой элементарных операций над расширенной матрицей системы, и после завершения шага алгоритма метода Гаусса-Жордана в матрице появляется единичный столбец. Затем шаги повторяются, но на очередном шаге запрещается выбирать разрешающий элемент в строках, в которых он уже выбирался на предыдущих шагах. Шаги продолжаются до тех пор, пока количество единичных столбцов не сравняется с количеством ненулевых строк расширенной матрицы. Мы получаем систему в базисном виде.

При работе методом Гаусса-Жордана возможны следующие две особые ситуации. В результате выполнения очередного шага могут появиться либо нулевая строка
$(0, 0,\dots, 0|0)$, либо строка вида $(0, 0,\dots, 0|b)$, где $b \neq 0$ . В первом случае в новой системе будет уравнение вида $0*x_1 + 0*x_2 + \dots + 0*x_n=0$, которое является тождеством, справедливым при любых значениях неизвестных. Отбрасывание этого уравнения не меняет множества решений системы, поэтому обычно нулевая строка отбрасывается, и работа алгоритма продолжается. Во втором случае в новой системе появляется уравнение $0*x_1 + 0*x_2 + \dots + 0*x_n \neq 0$, которое не может выполняться. Это свидетельствует о том, что новая и первоначальная системы несовместны. В этом случае работа алгоритма прекращается.

\textbf{Пример 1.5.}  Реализуем метод Гаусса-Жордана для системы
%ТУТ ЕСТЬ СЛАУ

В расширенной матрице системы выберем разрешающий элемент в первой строке и первом столбце. Получим сразу ведущую строку. Умножая первую строку на (-3) и прибавляя ко второй, получим
%ТУТ ЕСТЬ МАТРИЦА

Выбирая разрешающий элемент во второй строке и третьем столбце, умножая вторую строку на (-5) и прибавляя к первой, получим
%ТУТ ЕСТЬ МАТРИЦА

Последняя матрица соответствует системе уравнений
%ТУТ ЕСТЬ СЛАУ
с базисными неизвестными $x_1$, $x_3$ и свободной неизвестной $x_2$.

Отметим, что рассмотренный нами метод содержит большую долю произвола при выборе разрешающего элемента. Полученный в результате базисный вид системы тоже определяется неоднозначно. У совместной системы существует некоторая конечная совокупность базисных видов.

Переход от одного базисного вида к другому можно произвести с помощью, так называемой, \textit{операции замещения}. Эта операция переводит заданную базисную неизвестную $x_i$ в разряд свободных, а заданную свободную неизвестную $x_j$ — в базисную. Операция замещения состоит в дополнительном шаге алгоритма Гаусса-Жордана с особым выбором разрешающего элемента. Этот элемент выбирается в строке, содержащей единицу при базисной неизвестной $x_i$ и в столбце, отвечающем свободной неизвестной $x_j$. Выполним операцию замещения в базисном виде системы предыдущего примера, замещая свободной переменной  $x_2$ базисную $x_1$
%ТУТ ЕСТЬ МАТРИЦА
Тем самым получен новый базисный вид системы
%ТУТ ЕСТЬ СЛАУ

В следующем параграфе при изучении симплекс-метода мы встретимся с базисным видом линейной системы уравнений и операцией замещения. При этом неизвестные будут называться переменными.



\subsection{Симплекс-метод}
Этот метод является универсальным, применимым к любой задаче линейного программирования в канонической форме. Система ограничений здесь — система линейных уравнений, в которой количество неизвестных обычно  больше  количества  уравнений. Если  ранг  системы равен $r$, то мы можем выбрать $r$ неизвестных, которые выразим через все остальные неизвестные. Для определенности предположим, что выбраны  первые, идущие подряд, неизвестные $x_1,x_2,\dots,x_r$ . Тогда наша система уравнений может быть записана в виде
%ТУТ ЕСТЬ СЛАУ

К такому виду можно привести любую совместную систему, например методом Гаусса-Жордана. Правда, не всегда можно выражать через остальные первые $r$ неизвестных (мы это сделали для определенности записи). Однако такие $r$ неизвестных обязательно найдутся. Эти неизвестные (переменные) называются \textit{базисными}. Остальные переменные называются   \textit{свободными}. Придавая определенные значения свободным переменным и вычисляя значения базисных (выраженных через свободные), мы будем получать различные решения нашей системы ограничений. Таким образом, можно получить любое ее решение. Нас будут интересовать особые решения, которые  получаются, когда свободные переменные равны нулю. Такие решения  называются \textit{базисными}. Базисных решений столько же, сколько  различных  базисных видов у данной системы ограничений. Базисное решение называется \textit{допустимым базисным решением} или \textit{опорным решением}, если в нем значения переменных неотрицательны. Если в качестве базисных взяты переменные $x_1$,$x_2$,\dots,$x_r$, то решение \{$b'_1$,$b'_2$,\dots,$b'_r$,0,\dots,0\} будет опорным, если $b'_1 \leq 0$,    $b'_2 \leq 0$, \dots, $b'_r \leq 0$. Симплекс-метод основан на следующей теореме, которая называется \textit{фундаментальной теоремой симплекс-метода}. 

\textbf{Теорема 1.1.} \textit{Среди оптимальных планов задачи линейного программирования в канонической форме обязательно есть опорное решение ее системы ограничений. Если оптимальный план задачи единственен, то он совпадает с некоторым опорным  решением.}

Различных опорных решений системы ограничений конечное число. Поэтому решение задачи в канонической форме можно было бы искать  перебором опорных решений и выбором среди них того решения,  для  которого  значение $z$ самое большое. Но, во-первых, все опорные решения неизвестны, и их нужно находить, а во-вторых, в реальных задачах этих решений очень много, и прямой перебор вряд ли возможен. Симплекс-метод  представляет собой  некоторую  процедуру  направленного перебора опорных решений.  Исходя из некоторого, найденного заранее, опорного решения по определенному алгоритму симплекс-метода, мы подсчитываем новое опорное решение, на котором значение целевой функции $z$ не меньше, чем на старом. После ряда шагов мы приходим к опорному решению, которое является  оптимальным планом.
\begin{center}
\textit{\textbf{Процедура симплекс-метода на примере}}
\end{center}

Пусть требуется найти решение следующей задачи линейного программирования

%ТУТ СЛАУ
Первое опорное решение имеет следующий вид $B_1$=\{1; 2; 3; 0; 0\}. На этом базисном решении системы ограничений целевая функция $z(B_1)$ = 0. Из вида целевой функции заключаем, что она может быть увеличена при выходе из решения $B_1$   путем  увеличении переменной $x_5$ от нуля. При этом нужно следить, чтобы соблюдались равенства нашей системы и все переменные оставались неотрицательными. Из первого ограничения видно, что x1 останется неотрицательным при произвольном увеличении x5. Второе ограничение показывает, что $x_2$ становиться отрицательным при $x_5 > 2$. Из третьего ограничения заключаем, что $x_3$ остается неотрицательным при увеличении $x_5$ до трех. Таким образом, все переменные остаются неотрицательными при увеличении $x_5$ до 2. Предположим, что $x_5 =2$, но, по-прежнему, $x_4= 0$. Тогда остальные переменные примут значения $x_1= 5$; $x_2= 0$; $x_3 = 1$. Мы перешли к некоторому новому решению системы ограничений $B_2$=\{5; 0; 1; 0; 2\}. Это решение будет опорным решением и соответствующий базисный вид можно получить с помощью операции замещения, при которой x2 выводится из числа базисных, а $x_5$ становится базисной переменной. Другими словами, нужно $x_5$ выразить из второго равенства системы ограничений и полученное выражение подставить вместо $x_5$ в первое и третье равенства. В результате получим базисный вид системы
%ТУТ СЛАУ
которому отвечает базисное решение $B_2$=\{5; 0; 1; 0; 2\}. С помощью нового вида системы исключим $x_5$ из целевой функции задачи 
$$z = -x_4 + x_5 = 2 - x_2 + 2x_4 - x_4 = 2 - x_2 + x_4.$$
Значение целевой функции на новом базисном решении $z(B_2) = 2$. При этом целевую функцию можно еще увеличить, если выйти из $B_2$, увеличивая переменную $x_4$. В первом и во втором равенствах перестроенной системы ограничений $x_4$ можно увеличивать неограниченно. В третьем равенстве $x_4$ можно увеличивать лишь до 1/5. В противном случае переменная $x_3$ станет отрицательной. Положим $x_4 = 1/5$, $x_2 = 0$. Тогда $x_1 = 5 + 3/5 = 28/5$; $x_5 = 2 + 2 / 5 = 12 / 5$; $x_3 = 0$. Получаем следующее опорное решение $B_3=\{28/5; 0; 0; 1/5; 12/5\}$. При этом переменная $x_3$ должна быть выведена из состава базисных переменных, а вместо нее базисной переменной должна стать переменная x4. Производя операцию замещения, получаем следующий базисный вид системы ограничений
%ТУТ СЛАУ
с помощью которого можно исключить базисные переменные из выражения для целевой функции
$$z = 2 – x_2 + x_4 = 2 – x2 + ((1/5) + (1/5)x_2 – (1/5)x_3) = (11/5) – (4/5)x_2 – (1/5)x_3.$$
Из последнего выражения видно, что увеличить значение целевой функции переходом к новому опорному решению нельзя. Поэтому $z_{max}= z(B_3) = 11/5$. Оптимальный план совпадает с $B_3 = \{28/5; 0; 0; 1/5; 12/5\}$.
\begin{center}
\textit{\textbf{Процедура симплекс-метода в общем случае}}
\end{center}

В рассмотренном примере мы сделали два шага, переходя последовательно от базисного решения $B_1$ к $B_2$, а затем — к $B_3$. Вычисления по симплекс-методу обычно организуются в виде так называемых симплекс-таблиц. Чтобы разобраться в устройстве симплекс-таблицы рассмотрим один шаг симплекс-метода в общем случае. Предположим, что система ограничений задачи в канонической форме приведена к допустимому базисному виду и базисными переменными являются $x_1$, $x_2$, \dots, $x_r$. Целевая функция при этом выражена через свободные переменные $x_{r+1}$, \dots, $x_n$, то есть задача имеет вид
%ТУТ СЛАУ

Поскольку базисный вид является допустимым, то $b_1$, $b_2$, \dots, $b_r \leq 0$. Работа по симплекс-методу начинается с просмотра коэффициентов целевой функции, то есть величин $\gamma_{r+1}$, \dots, $\gamma_n$. Здесь могут представиться два случая.
		\begin{enumerate}
			\renewcommand{\theenumi}{(\arabic{enumi})}
			\renewcommand{\labelenumi}{\arabic{enumi})}
			\itemВсе коэффициенты неположительные: $\gamma_{r+1}$, \dots, $\gamma_n$. В этом случае исходное базисное решение будет оптимальным, так как переходом к другому базисному решению мы не можем увеличить целевую функцию.
			\itemСреди коэффициентов целевой функции есть положительные.
		\end{enumerate}

Пусть $\gamma_j > 0$. Это значит, что увеличение переменной $x_j$ ведет к увеличению целевой функции. При этом увеличивать $x_j$ нужно так, чтобы базисные переменные оставались неотрицательными. Для определения границы увеличения $x_j$ просмотрим столбец коэффициентов при $x_j$ в системе ограничений, то есть числа
$a_{1j}$, $a_{2j}$, \dots, $a_{rj}$. Если все эти коэффициенты отрицательны, то $x_j$ можно увеличивать неограниченно, сохраняя неотрицательными базисные переменные. Это означает, что целевая функция неограниченна на области допустимых значений и задача не имеет решений. Если же среди этих коэффициентов есть положительные, то в соответствующих уравнениях $x_j$ нельзя увеличивать неограниченно. Предположим, что мы придаем $x_j$ значение $\rho  (x_j = \rho)$, а остальные свободные переменные оставляем равными нулю. Система ограничений примет вид
%ТУТ СЛАУ
Значение $\rho$ не может быть произвольным в i-ом равенстве, если $a_{ij}>0$. При этом $\rho$ можно увеличивать в i-ом равенстве до величины $b_i/a_{ij}$. Таким образом, граница увеличения $\rho$ может быть найдена выбором среди чисел $a_{1j}$, $a_{2j}$, \dots, $a_{rj}$ положительных, а затем для них минимального значения отношения $b_i/a_{ij}$. Это отношение и даст нам границу увеличения $\rho$. Если минимальным будет отношение $b_{i_0}$/$a_{i_0j}$, то граница увеличения $\rho$ определяется $i_0$-ым равенством и можно положить $\rho = b_{i_0}/a_{i_0j}$. При этом $x_{i_0} = 0$.  При перестройке системы ограничений переменная $x_j$ должна быть введена в число базисных переменных, а переменная  должна стать свободной. Перестройка производится с помощью операции замещения, которая выполняется с помощью разрешающего элемента $a_{i_0j}$. Этот элемент принято называть разрешающим или генеральным элементом шага симплекс-метода. После операции однократного замещения мы получим новый базисный вид системы ограничений, новое опорное решение и новое выражение для целевой функции через новые свободные переменные. На этом шаг симплекс-метода завершается. Следующий шаг начинается с просмотра коэффициентов нового выражения целевой функции. Шаги симплекс-метода повторяются до тех пор, пока на очередном шаге все коэффициенты выражения целевой функции через свободные переменные не станут неположительными. В этом случае очередное опорное решение системы ограничений будет точкой максимума.

Вычисления по симплекс-методу организуются в виде симплекс-таблиц, которые являются сокращенной записью задачи линейного программирования в канонической форме. \textit{Перед составлением симплекс-таблицы задача должна быть преобразована. Система ограничений приводится к допустимому базисному виду, с помощью которого из целевой функции должны быть исключены базисные  переменные.} Вопрос об этих преобразованиях мы рассмотрим ниже. Сейчас же будем считать, что они уже выполнены и задача имеет вид
%ТУТ СЛАУ

Здесь для определенности записи считается, что в качестве базисных можно взять переменные $x_1$,$x_2$,\dots,$x_r$ и что при этом $b'_1 \leq 0$,$b'_2 \leq 0$,\dots,$b'_r \leq 0$. (соответствующее  базисное решение является опорным).

Для составления симплекс-таблицы во всех равенствах в условии  задачи члены, содержащие переменные, переносятся в левую часть, а свободные члены оставляются справа, т.е. задача записывается в виде следующей системы равенств:
%ТУТ СЛАУ

Далее эта система оформляется  в виде таблицы:
%ТУТ ТАБЛИЦА

Еще раз напомним, что названия базисных переменных здесь взяты лишь для определенности записи и в реальной таблице могут оказаться другими.
\begin{center}
\textit{\textbf{Порядок работы с симплекс-таблицей}}
\end{center}

Первая симплекс-таблица подвергается преобразованию, суть которого заключается в переходе к новому опорному решению. Порядок перехода к следующей таблице такой.
\begin{enumerate}
			\itemПросматривается последняя строка таблицы и среди коэффициентов этой строки (исключая $\gamma_o$) выбирается отрицательное число. Если такового нет, то исходное базисное решение является оптимальным и данная  таблица является последней. 
			\itemПросматривается столбец таблицы, отвечающий выбранному отрицательному коэффициенту в последней строке, и в этом столбце выбираются положительные коэффициенты. Если таковых нет, то целевая функция  неограниченна на области допустимых значений переменных, и задача решений не имеет. 
			\itemСреди отобранных коэффициентов столбца выбирается тот, для которого отношение соответствующего свободного члена, находящегося  в столбце свободных членов, к этому элементу, минимально. Этот коэффициент называется \textit{разрешающим} или \textit{генеральным элементом таблицы}. В дальнейшем базисная переменная, отвечающая строке разрешающего элемента, должна быть переведена в разряд свободных, а  свободная переменная, отвечающая столбцу разрешающего элемента, вводится в число базисных. 
			\itemСтроится новая таблица, содержащая новые названия базисных  переменных. Строка разрешающего элемента делится на этот элемент, и  полученная строка записывается в новую таблицу на то же место. В остальные клетки новой таблицы записываются результаты преобразования элементов  старой таблицы. Для этого умножают первую из заполненных строк (строку разрешающего элемента) на некоторые числа и  складывают ее со строками старой таблицы. Числа эти подбираются так, чтобы в  столбце разрешающего элемента получились  нули, кроме  клетки разрешающего элемента, в которой стоит единица. В результате получают новую  симплекс-таблицу, которая отвечает новому базисному решению.
			\item Теперь следует обратиться к пункту 2, т.е. просмотреть строку целевой функции и повторить все вышеперечисленное. Составление новых симплекс-таблиц производится до тех пор, пока все коэффициенты последней строки (кроме стоящего на месте $\gamma_0$) в очередной таблице не станут неотрицательными. После этого считается, что задача решена и по  последней  симплекс-таблице прочитывается ответ задачи. Максимальное значение $z_{max}$ целевой  функции стоит в первой клетке последней строки на месте $y_o$. Неотрицательные значения новых базисных переменных стоят в  остальных клетках столбца свободных членов. Остальные переменные в точке  максимума равны нулю. 
		\end{enumerate}

\textit{Замечание.} Изложенный алгоритм содержит возможность неопределенности при выборе разрешающего элемента. Могут появиться несколько столбцов, пригодных для его выбора. Да и в заданном столбце может быть несколько чисел, каждое из которых можно назвать разрешающим элементом. Для однозначной организации вычислений приходится вводить добавочные условные правила. При выборе столбца разрешающего элемента в последней строке симплекс-таблицы выбирается максимальный по модулю отрицательный коэффициент. Если есть несколько таких коэффициентов с одинаковым максимальным модулем, выбирается тот, что отвечает переменной с минимальным номером и т.д. Отметим также, что если в столбце, пригодном для выбора разрешающего элемента, нет положительных чисел, то задача не имеет решений по причине неограниченности целевой функции на области допустимых планов.

Рассмотрим порядок решения задачи с помощью симплекс-таблиц на примерах.

\textbf{Пример 1.6.} Решить следующую задачу, уже рассмотренную в качестве примера:
%ТУТ СЛАУ
Запишем задачу  в  виде равенств:
%ТУТ СЛАУ

Составляем первую симплекс-таблицу. Находим разрешающий элемент.
%ТУТ ТАБЛИЦА

Составляем новую симплекс-таблицу. Снова находим разрешающий элемент.
%ТУТ ТАБЛИЦА

Переходим к следующей таблице:
%ТУТ ТАБЛИЦА

Эта  таблица является последней, по ней читаем ответ задачи: $z_{max}$ = ${11}\over{5}$ Координаты точки максимума: $x_1$ = ${28}\over{5}$; $x_2 = 0$; $x_3 = 0$; $x_4$ = ${1}\over{5}$; $x_5$ = ${12}\over{5}$;

\textbf{Пример 1.7.} Решить задачу:
%ТУТ СЛАУ
Составляем первую симплекс-таблицу и находим разрешающий элемент.
%ТУТ ТАБЛИЦА
Вторая таблица имеет вид:

Поскольку в последней таблице в столбце, пригодном для выбора разрешающего элемента, нет положительных чисел, то целевая функция неограниченна на области допустимых значений, то есть задача решения не имеет.
\begin{center}
\textit{\textbf{Зацикливание симплекс алгоритма}}
\end{center}

При работе симплекс методом мы переходим от одного опорного решения системы ограничений задачи к другому, причем так, чтобы значение целевой функции на следующем решении не уменьшалось. Поскольку опорных решений конечное число, мы должны прийти в оптимальное опорное решение, существование которого гарантируется фундаментальной теоремой. Однако мы можем не достигнуть оптимального решения, если в процессе перебора симплекс методом вернемся в опорное решение, которое уже встречалось ранее, и затем будем повторять цепочку опорных решений, пройденных ранее. Вычисления по симплекс методу войдут в бесконечно повторяющийся цикл и никогда не закончатся. Такое явление называется \textit{зацикливанием симплекс алгоритма.}

Зацикливание — явление очень редкое. В литературе по линейному программированию имеются лишь несколько примеров задач, в которых возможно возникновение зацикливания.

Несмотря на возможность зацикливания, любую задачу линейного программирования надлежащей формы можно решить симплекс методом до конца. При рассмотрении симплекс алгоритма мы видели, что на очередном шаге разрешающий элемент может  выбираться неоднозначно. Для однозначной организации вычислений приходится вводить добавочные правила. Можно показать, что зацикливание наступает лишь в случае возможности неоднозначного выбора разрешающего элемента. Если зацикливание наступило, следует изменить порядок вычислений, выбирая разрешающий элемент по-другому. Произойдет выход из цикла. Для борьбы с зацикливанием используют особые подпрограммы, гарантирующие выход из цикла в случае наступления зацикливания.














\end{document}
