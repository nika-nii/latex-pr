\documentclass[12pt]{article}
\usepackage[russian]{babel}
\usepackage[utf8]{inputenc} % set input encoding (not needed with XeLaTeX)

\title{}
\begin{document}
\maketitle





\subsection{Общие системы линейных уравнений. Базисный вид системы. Метод Гаусса-Жордана}
Ниже нам понадобятся некоторые сведения о системах уравнений, состоящих из m линейных уравнений с \textit {n} неизвестными $x_1$,$x_2$,\dots,$x_\textit{n}$%
% ТУТ ЕСТЬ ФОРМУЛА

Такая система может иметь множество решений, состоящее из единственного решения или бесконечного количества решений. Возможен также случай \textit{несовместности системы}, когда множество решений является пустым. Две системы называются\textit {эквивалентными}, если они имеют одинаковые множества решений. Следующие операции над системой приводят к новой системе, эквивалентной исходной













\end{document}
