\documentclass{article}

\usepackage[russian]{babel}
\usepackage[utf8]{inputenc} 

\begin{document}






\subsection{Общие системы линейных уравнений. Базисный вид системы. Метод Гаусса-Жордана}
Ниже нам понадобятся некоторые сведения о системах уравнений, состоящих из m линейных уравнений с \textit {n} неизвестными $x_1$,$x_2$,\dots,$x_\textit{n}$%
% ТУТ ЕСТЬ СЛАУ

Такая система может иметь множество решений, состоящее из единственного решения или бесконечного количества решений. Возможен также случай \textit{несовместности системы}, когда множество решений является пустым. Две системы называются\textit {эквивалентными}, если они имеют одинаковые множества решений. Следующие операции над системой приводят к новой системе, эквивалентной исходной:
		\begin{enumerate}
			\renewcommand{\theenumi}{(\arabic{enumi})}
			\renewcommand{\labelenumi}{\arabic{enumi})}
			\itemперестановка уравнений системы;
			\itemумножение обеих частей уравнения системы на одно и то же отличное от нуля число;
			\itemприбавление к заданному уравнению системы любого другого уравнения, умноженного на произвольное число.
		\end{enumerate}
Эти операции называют элементарными операциями над системой.

Основной задачей теории линейных систем является задача нахождения всего множества решений системы или установления ее несовместности. Для ее решения нужно привести систему к эквивалентной системе, имеющей особый \textit{базисный вид}. При этом систему условно записывают в виде расширенной матрицы системы
%ТУТ ЕСТЬ МАТРИЦА
Система называется имеющей базисный вид, если среди столбцов коэффициентов при неизвестных в ее расширенной матрице имеется столько  различных  единичных  столбцов, сколько  ненулевых строк в этой матрице. Единичным столбцом мы называем столбец, в котором на некотором месте стоит единица, а на всех остальных местах — нули. Единичные столбцы считаются различными, если единицы у них находятся на различных местах. Неизвестные, отвечающие этим различным единичным столбцам, называются \textit{базисными неизвестными системы}. Остальные неизвестные называются \textit{свободными неизвестными}. Базисные неизвестные входят по одному в каждое из уравнений и легко выражаются через свободные неизвестные. Если свободным неизвестным придавать произвольные значения, то базисные неизвестные по ним определятся однозначно, и мы получим все решения системы.

\textbf{Пример 1.4.} Рассмотрим систему
%ТУТ ЕСТЬ СЛАУ

Ее расширенная матрица
%ТУТ ЕСТЬ МАТРИЦА

Система имеет базисный вид. Базисными неизвестными будут $x_1$ и $x_4$, а свободными — $x_2$ и $x_3$.
%ТУТ ЕСТЬ СЛАУ

Полагая, свободные неизвестные произвольными $x_2=c_1$, $x_3=c_2$, получаем двухпараметрическое семейство всех решений нашей системы ($6-3c_1-5c_2;c_1;c_2;-1-7c_1+2c_2$).

Свободные неизвестные могут отсутствовать в базисном виде системы. Тогда, очевидно, система имеет только одно решение.

Метод Гаусса-Жордана представляет собой некоторый алгоритм, приводящий систему к базисному виду с помощью цепочки элементарных преобразований, которую удобно выполнять не над системой, а над ее расширенной матрицей. При этом элементарные операции над системой становятся следующими операциями над расширенной матрицей:
		\begin{enumerate}
			\renewcommand{\theenumi}{(\arabic{enumi})}
			\renewcommand{\labelenumi}{\arabic{enumi})}
			\itemперестановка строк в матрице;
			\itemумножение всех элементов некоторой строки на одно и то же отличное от нуля число;
			\itemприбавление к данной строке любой другой, умноженной на произвольное число.
		\end{enumerate}

Метод Гаусса-Жордана состоит из ряда однотипных шагов. Опишем первый шаг алгоритма. Он состоит из трех этапов: 
		\begin{enumerate}
			\renewcommand{\theenumi}{(\arabic{enumi})}
			\renewcommand{\labelenumi}{\arabic{enumi})}
			\itemсреди коэффициентов при неизвестных расширенной матрицы системы выбирается отличное от нуля число, которое в дальнейшем мы называем  \textit{разрешающим элементом} шага метода;
			\itemстрока разрешающего элемента делится на разрешающий элемент и полученная строка, становясь основным инструментом для преобразования матрицы, называется нами в дальнейшем \textit{ведущей строкой} шага алгоритма;
			\itemведущая строка преобразует остальные строки матрицы путем прибавления ее к этим строкам после умножения на так подобранные числа, чтобы после преобразований в столбце бывшего разрешающего элемента стояли нули на всех местах, кроме места самого разрешающего элемента (на котором находится единица).
		\end{enumerate}

Описанные преобразования являются цепочкой элементарных операций над расширенной матрицей системы, и после завершения шага алгоритма метода Гаусса-Жордана в матрице появляется единичный столбец. Затем шаги повторяются, но на очередном шаге запрещается выбирать разрешающий элемент в строках, в которых он уже выбирался на предыдущих шагах. Шаги продолжаются до тех пор, пока количество единичных столбцов не сравняется с количеством ненулевых строк расширенной матрицы. Мы получаем систему в базисном виде.

При работе методом Гаусса-Жордана возможны следующие две особые ситуации. В результате выполнения очередного шага могут появиться либо нулевая строка
$(0, 0,\dots, 0|0)$, либо строка вида $(0, 0,\dots, 0|b)$, где $b \neq 0$ . В первом случае в новой системе будет уравнение вида $0*x_1 + 0*x_2 + \dots + 0*x_n=0$, которое является тождеством, справедливым при любых значениях неизвестных. Отбрасывание этого уравнения не меняет множества решений системы, поэтому обычно нулевая строка отбрасывается, и работа алгоритма продолжается. Во втором случае в новой системе появляется уравнение $0*x_1 + 0*x_2 + \dots + 0*x_n \neq 0$, которое не может выполняться. Это свидетельствует о том, что новая и первоначальная системы несовместны. В этом случае работа алгоритма прекращается.

\textbf{Пример 1.5.}  Реализуем метод Гаусса-Жордана для системы
%ТУТ ЕСТЬ СЛАУ

В расширенной матрице системы выберем разрешающий элемент в первой строке и первом столбце. Получим сразу ведущую строку. Умножая первую строку на (-3) и прибавляя ко второй, получим
%ТУТ ЕСТЬ МАТРИЦА

Выбирая разрешающий элемент во второй строке и третьем столбце, умножая вторую строку на (-5) и прибавляя к первой, получим
%ТУТ ЕСТЬ МАТРИЦА

Последняя матрица соответствует системе уравнений
%ТУТ ЕСТЬ СЛАУ
с базисными неизвестными $x_1$, $x_3$ и свободной неизвестной $x_2$.

Отметим, что рассмотренный нами метод содержит большую долю произвола при выборе разрешающего элемента. Полученный в результате базисный вид системы тоже определяется неоднозначно. У совместной системы существует некоторая конечная совокупность базисных видов.

Переход от одного базисного вида к другому можно произвести с помощью, так называемой, \textit{операции замещения}. Эта операция переводит заданную базисную неизвестную $x_i$ в разряд свободных, а заданную свободную неизвестную $x_j$ — в базисную. Операция замещения состоит в дополнительном шаге алгоритма Гаусса-Жордана с особым выбором разрешающего элемента. Этот элемент выбирается в строке, содержащей единицу при базисной неизвестной $x_i$ и в столбце, отвечающем свободной неизвестной $x_j$. Выполним операцию замещения в базисном виде системы предыдущего примера, замещая свободной переменной  $x_2$ базисную $x_1$
%ТУТ ЕСТЬ МАТРИЦА
Тем самым получен новый базисный вид системы
%ТУТ ЕСТЬ СЛАУ

В следующем параграфе при изучении симплекс-метода мы встретимся с базисным видом линейной системы уравнений и операцией замещения. При этом неизвестные будут называться переменными.


\end{document}
